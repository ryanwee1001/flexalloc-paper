%%%%%%%%%%%%%%%%%%%%%%%%%%%%%%%%%%%%%%%%%%%%%%%%%%%%%%%%%%%%%%%%%%%%%%%%%%%%%%%%
% Template for USENIX papers.
%
% History:
%
% - TEMPLATE for Usenix papers, specifically to meet requirements of
%   USENIX '05. originally a template for producing IEEE-format
%   articles using LaTeX. written by Matthew Ward, CS Department,
%   Worcester Polytechnic Institute. adapted by David Beazley for his
%   excellent SWIG paper in Proceedings, Tcl 96. turned into a
%   smartass generic template by De Clarke, with thanks to both the
%   above pioneers. Use at your own risk. Complaints to /dev/null.
%   Make it two column with no page numbering, default is 10 point.
%
% - Munged by Fred Douglis <douglis@research.att.com> 10/97 to
%   separate the .sty file from the LaTeX source template, so that
%   people can more easily include the .sty file into an existing
%   document. Also changed to more closely follow the style guidelines
%   as represented by the Word sample file.
%
% - Note that since 2010, USENIX does not require endnotes. If you
%   want foot of page notes, don't include the endnotes package in the
%   usepackage command, below.
% - This version uses the latex2e styles, not the very ancient 2.09
%   stuff.
%
% - Updated July 2018: Text block size changed from 6.5" to 7"
%
% - Updated Dec 2018 for ATC'19:
%
%   * Revised text to pass HotCRP's auto-formatting check, with
%     hotcrp.settings.submission_form.body_font_size=10pt, and
%     hotcrp.settings.submission_form.line_height=12pt
%
%   * Switched from \endnote-s to \footnote-s to match Usenix's policy.
%
%   * \section* => \begin{abstract} ... \end{abstract}
%
%   * Make template self-contained in terms of bibtex entires, to allow
%     this file to be compiled. (And changing refs style to 'plain'.)
%
%   * Make template self-contained in terms of figures, to
%     allow this file to be compiled. 
%
%   * Added packages for hyperref, embedding fonts, and improving
%     appearance.
%   
%   * Removed outdated text.
%
%%%%%%%%%%%%%%%%%%%%%%%%%%%%%%%%%%%%%%%%%%%%%%%%%%%%%%%%%%%%%%%%%%%%%%%%%%%%%%%%

\documentclass[letterpaper,twocolumn,10pt]{article}
\usepackage{usenix}

% to be able to draw some self-contained figs
\usepackage{tikz}
\usepackage{amsmath}

% inlined bib file
\usepackage{filecontents}

%-------------------------------------------------------------------------------
\begin{filecontents}{\jobname.bib}
%-------------------------------------------------------------------------------
@InProceedings{hypsec,
  author =       {Li, Shih-Wei and Koh, John S. and Nieh, Jason},
  title =        {Protecting Cloud Virtual Machines from Commodity Hypervisor and Host Operating System Exploit},
  booktitle =    {Proceedings of the 28th USENIX Security Symposium},
  year =         2019,
  pages =        {1357--1374},
  note =         {\url{https://www.cs.columbia.edu/~nieh/pubs/security2019_hypsec.pdf}}
}
@InProceedings{sekvm,
  author =       {Li, Shih-Wei and Li, Xupeng and Gu, Ronghui and Nieh, Jason and Hui, John Z.},
  title =        {Formally Verified Memory Protection for a Commodity Multiprocessor Hypervisor},
  booktitle =    {Proceedings of the 30th USENIX Security Symposium.},
  year =         2021,
  pages =        {3953--3970},
  note =         {\url{https://www.usenix.org/system/files/sec21-li-shih-wei.pdf}}
}
@Misc{xuheng,
    author =        {Li, Xuheng},
    title =         {\text{Fall 23 Operating Systems 2 Homework}},
    year =          {2023},
    howpublished =  {\url{https://xuhengli.notion.site}},
    note =          {Accessed: 2023-12-21}
}
@Misc{arm,
    author =        {ARM Developer Documentation},
    title =         {Learn the architecture - AArch64 virtualization},
    howpublished =  {\url{https://developer.arm.com/documentation/102142/0100/Stage-2-translation}},
    note =          {Accessed: 2023-12-21}
}
@Misc{vmlinux.lds.S,
    author =        {XuhengLi (GitHub username)},
    title =         {vmlinux.lds.S in osdi23-paper114-sekvm},
    howpublished =  {\url{https://github.com/columbia/osdi23-paper114-sekvm/blob/ae/arch/arm64/kernel/vmlinux.lds.S}},
    note =          {Accessed: 2023-12-21}
}
@Misc{kernel-pgtable.h,
    author =        {XuhengLi (GitHub username)},
    title =         {kernel-pagetable.h in osdi23-paper114-sekvm},
    howpublished =  {\url{https://github.com/columbia/osdi23-paper114-sekvm/blob/ae/arch/arm64/include/asm/kernel-pgtable.h}},
    note =          {Accessed: 2023-12-21}
}
@Misc{el1.c,
    author =        {XuhengLi (GitHub username)},
    title =         {el1.c in osdi23-paper114-sekvm},
    howpublished =  {\url{https://github.com/columbia/osdi23-paper114-sekvm/blob/ae/arch/arm64/hypsec_proved/el1.c}},
    note =          {Accessed: 2023-12-21}       
}
\end{filecontents}

% %-------------------------------------------------------------------------------
\begin{document}
% %-------------------------------------------------------------------------------

%don't want date printed
\date{}

% make title bold and 14 pt font (Latex default is non-bold, 16 pt)
\title{\Large \bf FlexAlloc: Dynamic Memory Partitioning\\
    between the Corevisor and Hostvisor in SeKVM}

%for single author (just remove % characters)
\author{
{\rm Matthew Roger Nelson}\\
Columbia University
\and
{\rm Ryan Wee}\\
Columbia University
\and
{\rm Zeheng Yang}\\
Columbia University
} % end author

\maketitle

%-------------------------------------------------------------------------------
\begin{abstract}
%-------------------------------------------------------------------------------
SeKVM is a hypervisor for ARM processors that protects the memory of guest
virtual machines (VMs) from an untrusted host machine. It comprises a
corevisor that runs at EL2, and a hostvisor that runs at EL1
to leverage the functionality of the untrusted host kernel. However, SeKVM
statically partitions memory between the corevisor and hostvisor at compile
time. This leads to memory wastage when the number of VMs is low, and an
inability to support the desired number of VMs when the number of VMs is high.
In this paper, we present FlexAlloc, an extension of SeKVM that allows dynamic
runtime partitioning of memory between the corevisor and the hostvisor. In
particular, the memory needed for the stage-2 page tables of a guest VM is
only allocated to the corevisor when a VM is created. This memory is then
returned to the hostvisor when a VM is destroyed. Existing implementions
of \textit{alloc\_pages} do not support allocating a contiguous region of memory
equal in size to the memory needed for a VM's stage-2 page tables. Hence,
FlexAlloc also modifies the iterators used to accesss a VM's stage-2 page
tables, so that the memory used for a VM's stage-2 page tables can be composed
of noncontiguous regions. Our work ensures that a machine running SeKVM uses
memory more efficiently, and also allows such machines to support a larger
number of guest VMs. This allows public cloud providers and organizations
running their own VMs to scale up their operations at a lower cost.
\end{abstract}

%-------------------------------------------------------------------------------
\section{Introduction and Background}
%-------------------------------------------------------------------------------

Virtual machines (VMs) have become increasingly popular over the past few
decades. In particular, they improve utilization of existing hardware by
allowing users to run multiple unmodified operating systems on a single machine.
A key concern when it comes to VMs is isolation. A single machine may be used to
run multiple VMs created by different users who generally want their data
to be protected from one another. As a result, hypervisors generally maintain
some sort of secure boundary between guest VMs. However, another important
aspect of isolation is protecting guest VMs from the host machine. A hypervisor
that fully trusts its host machine is vulnerable to malicious actors who
gain control of the host operating system itself. This is dangerous because
the host machine could be used to steal confidential corporate data or sensitive
personal information.

SeKVM is a formally verified hypervisor for ARM that isolates KVM's trusted computing
base into a small core~\cite{sekvm}. It builds on the previous work of HypSec,
which is a hypervisor that protects VMs from the host machine~\cite{hypsec}.
HypSec consists of two layers. The first layer is an untrusted
\textit{hostvisor}, which operates at EL1 together with the host operating
system. The hostvisor leverages many of the data structures and functions
built into the host operating system for VM management. The second layer is a
trusted \textit{corevisor}, which operates at the more privileged EL2 level.
This corevisor is responsible for isolating VMs from each other and from the
host machine. SeKVM builds on HypSec's design, with the main difference being
that its trusted computing base is formally verified. In SeKVM, the two layers
are termed \textit{KServ} and \textit{KCore} respectively. However, in this
paper, we use the terms hostvisor and corevisor to remain
consistent with the underlying HypSec kernel source code.

SeKVM protects guest VMs in three main areas: CPU, memory, and I/O. In the
area of memory, SeKVM leverages ARM hardware support for virtualization. ARM
allows hypervisors to control memory access using stage-2
translation~\cite{arm}. In particular, the guest operating systems and host
operating system each think they are using page tables to map a virtual address space to the
physical address space. However, in reality, they only control the mapping
of virtual addresses to intermediate physical addresses. When translating
virtual addresses, the MMU starts by using these page tables to translate
virtual addresses into intermediate physical addresses. However, it then 
uses a second set of page tables to translate
intermediate phyiscal addresses into the actual machine physical addresses.
This second set of page tables, called stage-2 page tables, are controlled by
the hypervisor running at EL2. In the context of SeKVM, the stage-2 page tables
are managed by the corevisor. When the host operating system tries to
access some physical memory, the corevisor first checks that this
physical memory does not belong to a guest VM.

The hostvisor cannot be allowed to access the memory used to store
the stage-2 page tables of guest VMs. Otherwise, it could simply modify these
stage-2 page tables so that guest VMs write their data to memory that can be
accessed and manipulated by the hostvisor. In other words, the memory used for
these stage-2 page tables must `belong' to the corevisor. In SeKVM,
the corevisor is statically allocated this memory on bootup. This is done by
a linker script, \textit{arch/arm64/kernel/vmlinux.lds.S}~\cite{vmlinux.lds.S}, which allocates some
contiguous region of memory for the stage-2 page tables and demarcates it using the labels \textit{stage2\_pgs\_start}
and \textit{stage2\_pgs\_end}. The size of this memory region is
determined by the macro \textit{STAGE2\_PAGES\_SIZE}, which is defined in
\textit{arch/arm64/include/asm/kernel-pgtable.h}~\cite{kernel-pgtable.h}. In
particular, this memory region is large enough to store the stage-2 page table
for the corevisor, the stage-2 page table for the hostvisor,
and the stage-2 page tables for sixteen guest VMs. The corevisor and hostvisor are
each allocated sixteen 2M pages for their stage-2 page tables, while each guest VM
is allocated four 2M pages. Altogether, the corevisor is given 96 2M
pages for all of these stage-2 page tables. During bootup, the hostvisor
allocates the beginning of this memory region to the corevisor and itself. It then 
initializes the corevisor's VM-management data structure such that the remainder of this stage-2 page table memory region
is evenly divided between each of the sixteen possible guest VMs. In particular, the hostvisor sets
\textit{el2\_data->vm\_info[i].page\_pool\_start =
pool\_start + (STAGE2\_VM\_POOL\_SIZE * (i - 1))} for each possible VMID~\cite{el1.c}.
Here, \textit{vm\_info[i].page\_pool\_start} marks the start of the stage-2
page table for the \textit{i}th VM.

The key observation here is that this partitioning of memory is done statically
at compile time. In other words, the hostvisor cannot reclaim the memory it has
allocated to the corevisor. Similarly, the corevisor cannot ask for more memory
from the hostvisor. This is in contrast to the dynamic partitioning of memory
between the hostvisor and guest VMs. Guest VMs can request memory from the
hostvisor via a page fault, and the hostvisor can reclaim memory from guest VMs
using ballooning~\cite{hypsec}. The static partitioning of memory between the
hostvisor and the corevisor is undesirable for two reasons:

\begin{itemize}
    \item First, unused memory cannot be reclaimed from the corevisor. Say
    the number of VMs being run is $n$, such that $n < 16$. Then there will be
    $(16 - n) * 4$ 2M pages reserved for the stage-2 page tables of guest VMs that
    do not exist. Dynamic partitioning of memory would allow this memory to be reclaimed by the hostvisor, to
    alleviate memory pressure in the host. This memory could also be given to guest VMs, to
    alleviate memory pressure in these VMs. Giving VMs more memory would improve
    VM performance, because less data would need to be moved to the swap partition. 
    \item Second, there is a hard upper limit on the number of VMs that can
    be supported by SeKVM. Users cannot add VMs above the limit of sixteen,
    because there will be no memory in the corevisor to support the stage-2
    page tables of these VMs. Users could always choose to statically allocate
    even more memory to the corevisor upon bootup, but this would exacerbate
    the problem of unused memory mentioned above.
\end{itemize}

Static allocation of memory to the corevisor is not limited to memory for the
stage-2 page tables of each VM. In addition to this, SeKVM also statically
allocates memory to the corevisor to store the metadata of each VM. The same linker script
mentioned above reserves another contiguous region of memory equal in size to 32
2M pages, and demarcates it using \textit{el2\_data\_start}
and \textit{el2\_data\_end}~\cite{vmlinux.lds.S}. This memory is given to the
corevisor on bootup. Part of this memory is used to
store an array of \textit{struct el2\_vm\_info} objects, where each object is
used to store the metadata of each guest VM. SeKVM could potentially also benefit
from dynamic allocation of the memory used to store this VM metadata, where
this memory is only given to the corevisor upon VM creation and returned to the
hostvisor upon VM deletion. However,
we chose not to focus on this because the size of each \textit{struct el2\_vm\_info} object
is 768 bytes. The total size of all of the \textit{struct el2\_vm\_info} objects
is 12288 bytes, which is less than four 4K pages in total. As a result, the
problem of unused memory is not as significant, since we waste at most four
4K pages of memory. In any case, memory can only be allocated at the
page granularity. Hence, in the case where we want to create a new VM and there is no 
more space in the existing dynamically-allocated memory for the new \textit{struct el2\_vm\_info} object, we would
inevitably end up allocating an entirely new 4K page, the majority of which would remain unused. (This
is not the case with VM stage-2 page tables, since the size of the stage-2 page
table for each VM is nicely page-aligned.) The problem of a hard upper limit
is also not as significant, since we could easily make the array larger without
incurring much memory overhead.

%-------------------------------------------------------------------------------
\section{Design}
%-------------------------------------------------------------------------------

TODO

%-------------------------------------------------------------------------------
\section{Implementation}
%-------------------------------------------------------------------------------

TODO

%-------------------------------------------------------------------------------
\section{Evaluation}
%-------------------------------------------------------------------------------

TODO

%-------------------------------------------------------------------------------
\section{Future Work}
%-------------------------------------------------------------------------------

TODO

%-------------------------------------------------------------------------------
\section{Conclusions}
%-------------------------------------------------------------------------------

TODO

%-------------------------------------------------------------------------------
% \section{Footnotes, Verbatim, and Citations}
%-------------------------------------------------------------------------------

% Footnotes should be places after punctuation characters, without any
% spaces between said characters and footnotes, like so.
% \footnote{Remember that USENIX format stopped using endnotes and is
%   now using regular footnotes.} And some embedded literal code may
% look as follows.

% \begin{verbatim}
% int main(int argc, char *argv[]) 
% {
%     return 0;
% }
% \end{verbatim}

% Now we're going to cite somebody. Watch for the cite tag. Here it
% comes. Arpachi-Dusseau and Arpachi-Dusseau co-authored an excellent OS
% book, which is also really funny~\cite{arpachiDusseau18:osbook}, and
% Waldspurger got into the SIGOPS hall-of-fame due to his seminal paper
% about resource management in the ESX hypervisor~\cite{waldspurger02}.

% The tilde character (\~{}) in the tex source means a non-breaking
% space. This way, your reference will always be attached to the word
% that preceded it, instead of going to the next line.

% And the 'cite' package sorts your citations by their numerical order
% of the corresponding references at the end of the paper, ridding you
% from the need to notice that, e.g, ``Waldspurger'' appears after
% ``Arpachi-Dusseau'' when sorting references
% alphabetically~\cite{sekvm,hypsec}. 

% It'd be nice and thoughtful of you to include a suitable link in each
% and every bibtex entry that you use in your submission, to allow
% reviewers (and other readers) to easily get to the cited work, as is
% done in all entries found in the References section of this document.

% Now we're going take a look at Section~\ref{sec:figs}, but not before
% observing that refs to sections and citations and such are colored and
% clickable in the PDF because of the packages we've included.

%-------------------------------------------------------------------------------
% \section{Floating Figures and Lists}
% \label{sec:figs}
%-------------------------------------------------------------------------------


%---------------------------
% \begin{figure}
% \begin{center}
% \begin{tikzpicture}
%   \draw[thin,gray!40] (-2,-2) grid (2,2);
%   \draw[<->] (-2,0)--(2,0) node[right]{$x$};
%   \draw[<->] (0,-2)--(0,2) node[above]{$y$};
%   \draw[line width=2pt,blue,-stealth](0,0)--(1,1)
%         node[anchor=south west]{$\boldsymbol{u}$};
%   \draw[line width=2pt,red,-stealth](0,0)--(-1,-1)
%         node[anchor=north east]{$\boldsymbol{-u}$};
% \end{tikzpicture}
% \end{center}
% \caption{\label{fig:vectors} Text size inside figure should be as big as
%   caption's text. Text size inside figure should be as big as
%   caption's text. Text size inside figure should be as big as
%   caption's text. Text size inside figure should be as big as
%   caption's text. Text size inside figure should be as big as
%   caption's text. }
% \end{figure}
%% %---------------------------


% Here's a typical reference to a floating figure:
% Figure~\ref{fig:vectors}. Floats should usually be placed where latex
% wants then. Figure\ref{fig:vectors} is centered, and has a caption
% that instructs you to make sure that the size of the text within the
% figures that you use is as big as (or bigger than) the size of the
% text in the caption of the figures. Please do. Really.

% In our case, we've explicitly drawn the figure inlined in latex, to
% allow this tex file to cleanly compile. But usually, your figures will
% reside in some file.pdf, and you'd include them in your document
% with, say, \textbackslash{}includegraphics.

% Lists are sometimes quite handy. If you want to itemize things, feel
% free:

% \begin{description}
  
% \item[fread] a function that reads from a \texttt{stream} into the
%   array \texttt{ptr} at most \texttt{nobj} objects of size
%   \texttt{size}, returning returns the number of objects read.

% \item[Fred] a person's name, e.g., there once was a dude named Fred
%   who separated usenix.sty from this file to allow for easy
%   inclusion.
% \end{description}

% \noindent
% The noindent at the start of this paragraph in its tex version makes
% it clear that it's a continuation of the preceding paragraph, as
% opposed to a new paragraph in its own right.


% \subsection{LaTeX-ing Your TeX File}
%-----------------------------------

% People often use \texttt{pdflatex} these days for creating pdf-s from
% tex files via the shell. And \texttt{bibtex}, of course. Works for us.

%-------------------------------------------------------------------------------
\section*{Acknowledgments}
%-------------------------------------------------------------------------------

Shih-Wei Li, John S. Koh, and Jason Nieh implemented HypSec, which serves as
the predecessor to SeKVM. Shih-Wei Li, Xupeng Li, Ronghui Gu, Jason Nieh, and
John Zhuang Hui implemented and formally verfiied SeKVM. Xuheng Li wrote a
helpful set of notes describing the boot and initialization process in SeKVM,
as well as the steps needed to build and run SeKVM~\cite{xuheng}. Jason Nieh and
Xuheng Li provided helpful comments on the project and helped with the design
on FlexAlloc. This paper was written as part of the class COMS E6118 Operating
Systems II at Columbia University.

%-------------------------------------------------------------------------------
\bibliographystyle{plain}
\bibliography{\jobname}

%%%%%%%%%%%%%%%%%%%%%%%%%%%%%%%%%%%%%%%%%%%%%%%%%%%%%%%%%%%%%%%%%%%%%%%%%%%%%%%%
\end{document}
%%%%%%%%%%%%%%%%%%%%%%%%%%%%%%%%%%%%%%%%%%%%%%%%%%%%%%%%%%%%%%%%%%%%%%%%%%%%%%%%

%%  LocalWords:  endnotes includegraphics fread ptr nobj noindent
%%  LocalWords:  pdflatex acks
