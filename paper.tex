%%%%%%%%%%%%%%%%%%%%%%%%%%%%%%%%%%%%%%%%%%%%%%%%%%%%%%%%%%%%%%%%%%%%%%%%%%%%%%%%
% Template for USENIX papers.
%
% History:
%
% - TEMPLATE for Usenix papers, specifically to meet requirements of
%   USENIX '05. originally a template for producing IEEE-format
%   articles using LaTeX. written by Matthew Ward, CS Department,
%   Worcester Polytechnic Institute. adapted by David Beazley for his
%   excellent SWIG paper in Proceedings, Tcl 96. turned into a
%   smartass generic template by De Clarke, with thanks to both the
%   above pioneers. Use at your own risk. Complaints to /dev/null.
%   Make it two column with no page numbering, default is 10 point.
%
% - Munged by Fred Douglis <douglis@research.att.com> 10/97 to
%   separate the .sty file from the LaTeX source template, so that
%   people can more easily include the .sty file into an existing
%   document. Also changed to more closely follow the style guidelines
%   as represented by the Word sample file.
%
% - Note that since 2010, USENIX does not require endnotes. If you
%   want foot of page notes, don't include the endnotes package in the
%   usepackage command, below.
% - This version uses the latex2e styles, not the very ancient 2.09
%   stuff.
%
% - Updated July 2018: Text block size changed from 6.5" to 7"
%
% - Updated Dec 2018 for ATC'19:
%
%   * Revised text to pass HotCRP's auto-formatting check, with
%     hotcrp.settings.submission_form.body_font_size=10pt, and
%     hotcrp.settings.submission_form.line_height=12pt
%
%   * Switched from \endnote-s to \footnote-s to match Usenix's policy.
%
%   * \section* => \begin{abstract} ... \end{abstract}
%
%   * Make template self-contained in terms of bibtex entires, to allow
%     this file to be compiled. (And changing refs style to 'plain'.)
%
%   * Make template self-contained in terms of figures, to
%     allow this file to be compiled. 
%
%   * Added packages for hyperref, embedding fonts, and improving
%     appearance.
%   
%   * Removed outdated text.
%
%%%%%%%%%%%%%%%%%%%%%%%%%%%%%%%%%%%%%%%%%%%%%%%%%%%%%%%%%%%%%%%%%%%%%%%%%%%%%%%%

\documentclass[letterpaper,twocolumn,10pt]{article}
\usepackage{usenix}

% to be able to draw some self-contained figs
\usepackage{tikz}
\usepackage{amsmath}

% inlined bib file
\usepackage{filecontents}

%-------------------------------------------------------------------------------
\begin{filecontents}{\jobname.bib}
%-------------------------------------------------------------------------------
@InProceedings{hypsec,
  author =       {Li, Shih-Wei and Koh, John S. and Nieh, Jason},
  title =        {Protecting Cloud Virtual Machines from Commodity Hypervisor and Host Operating System Exploit},
  booktitle =    {Proceedings of the 28th USENIX Security Symposium},
  year =         2019,
  pages =        {1357--1374},
  note =         {\url{https://www.cs.columbia.edu/~nieh/pubs/security2019_hypsec.pdf}}
}
@InProceedings{sekvm,
  author =       {Li, Shih-Wei and Li, Xupeng and Gu, Ronghui and Nieh, Jason and Hui, John Z.},
  title =        {Formally Verified Memory Protection for a Commodity Multiprocessor Hypervisor},
  booktitle =    {Proceedings of the 30th USENIX Security Symposium.},
  year =         2021,
  pages =        {3953--3970},
  note =         {\url{https://www.usenix.org/system/files/sec21-li-shih-wei.pdf}}
}
@Unpublished{xuheng,
    author =     {Li, Xuheng},
    title =      {Fall 23 Operating Systems 2 Homework},
    year =       {2023},
    note =       {\url{https://xuhengli.notion.site}}
}
\end{filecontents}

% %-------------------------------------------------------------------------------
\begin{document}
% %-------------------------------------------------------------------------------

%don't want date printed
\date{}

% make title bold and 14 pt font (Latex default is non-bold, 16 pt)
\title{\Large \bf FlexAlloc: Dynamic Memory Partitioning\\
    between the Corevisor and Hostvisor in SeKVM}

%for single author (just remove % characters)
\author{
{\rm Matthew Roger Nelson}\\
Columbia University
\and
{\rm Ryan Wee}\\
Columbia University
\and
{\rm Zeheng Yang}\\
Columbia University
} % end author

\maketitle

%-------------------------------------------------------------------------------
\begin{abstract}
%-------------------------------------------------------------------------------
SeKVM is a hypervisor for ARM processors that protects the memory of guest
virtual machines (VMs) from an untrusted host machine. It comprises a
\textit{corevisor} that runs at EL2, and a \textit{hostvisor} that runs at EL1
to leverage the functionality of the untrusted host kernel. However, memory is
statically partitioned at compile time between the corevisor and the hostvisor
in SeKVM. This leads to memory wastage when the number of VMs is low, and an
inability to support the desired number of VMs when the number of VMs is high.
In this paper, we present FlexAlloc, an extension of SeKVM that allows dynamic
runtime partitioning of memory between the corevisor and the hostvisor. In
particular, the memory needed for the stage-2 page tables of a guest VM is
only allocated to the corevisor when a VM is created. This memory is also
returned to the hostvisor when a VM is destroyed. Existing implementions
of \textit{alloc\_pages} do not support allocating a contiguous region of memory
equal in size to the memory needed for a VM's stage-2 page tables. Hence,
FlexAlloc also modifies the iterators used to accesss a VM's stage-2 page
tables, so that the memory used for a VM's stage-2 page tables can be composed
of noncontiguous regions. Our work ensures that a machine running SeKVM uses
memory more efficiently, and also allows such machines to support a larger
number of guest VMs. This allows public cloud providers and organizations
running their own VMs to scale up their operations at a lower cost.
\end{abstract}


%-------------------------------------------------------------------------------
\section{Introduction}
%-------------------------------------------------------------------------------

TODO~\cite{sekvm, hypsec}

%-------------------------------------------------------------------------------
% \section{Footnotes, Verbatim, and Citations}
%-------------------------------------------------------------------------------

% Footnotes should be places after punctuation characters, without any
% spaces between said characters and footnotes, like so.
% \footnote{Remember that USENIX format stopped using endnotes and is
%   now using regular footnotes.} And some embedded literal code may
% look as follows.

% \begin{verbatim}
% int main(int argc, char *argv[]) 
% {
%     return 0;
% }
% \end{verbatim}

% Now we're going to cite somebody. Watch for the cite tag. Here it
% comes. Arpachi-Dusseau and Arpachi-Dusseau co-authored an excellent OS
% book, which is also really funny~\cite{arpachiDusseau18:osbook}, and
% Waldspurger got into the SIGOPS hall-of-fame due to his seminal paper
% about resource management in the ESX hypervisor~\cite{waldspurger02}.

% The tilde character (\~{}) in the tex source means a non-breaking
% space. This way, your reference will always be attached to the word
% that preceded it, instead of going to the next line.

% And the 'cite' package sorts your citations by their numerical order
% of the corresponding references at the end of the paper, ridding you
% from the need to notice that, e.g, ``Waldspurger'' appears after
% ``Arpachi-Dusseau'' when sorting references
% alphabetically~\cite{sekvm,hypsec}. 

% It'd be nice and thoughtful of you to include a suitable link in each
% and every bibtex entry that you use in your submission, to allow
% reviewers (and other readers) to easily get to the cited work, as is
% done in all entries found in the References section of this document.

% Now we're going take a look at Section~\ref{sec:figs}, but not before
% observing that refs to sections and citations and such are colored and
% clickable in the PDF because of the packages we've included.

%-------------------------------------------------------------------------------
% \section{Floating Figures and Lists}
% \label{sec:figs}
%-------------------------------------------------------------------------------


%---------------------------
% \begin{figure}
% \begin{center}
% \begin{tikzpicture}
%   \draw[thin,gray!40] (-2,-2) grid (2,2);
%   \draw[<->] (-2,0)--(2,0) node[right]{$x$};
%   \draw[<->] (0,-2)--(0,2) node[above]{$y$};
%   \draw[line width=2pt,blue,-stealth](0,0)--(1,1)
%         node[anchor=south west]{$\boldsymbol{u}$};
%   \draw[line width=2pt,red,-stealth](0,0)--(-1,-1)
%         node[anchor=north east]{$\boldsymbol{-u}$};
% \end{tikzpicture}
% \end{center}
% \caption{\label{fig:vectors} Text size inside figure should be as big as
%   caption's text. Text size inside figure should be as big as
%   caption's text. Text size inside figure should be as big as
%   caption's text. Text size inside figure should be as big as
%   caption's text. Text size inside figure should be as big as
%   caption's text. }
% \end{figure}
%% %---------------------------


% Here's a typical reference to a floating figure:
% Figure~\ref{fig:vectors}. Floats should usually be placed where latex
% wants then. Figure\ref{fig:vectors} is centered, and has a caption
% that instructs you to make sure that the size of the text within the
% figures that you use is as big as (or bigger than) the size of the
% text in the caption of the figures. Please do. Really.

% In our case, we've explicitly drawn the figure inlined in latex, to
% allow this tex file to cleanly compile. But usually, your figures will
% reside in some file.pdf, and you'd include them in your document
% with, say, \textbackslash{}includegraphics.

% Lists are sometimes quite handy. If you want to itemize things, feel
% free:

% \begin{description}
  
% \item[fread] a function that reads from a \texttt{stream} into the
%   array \texttt{ptr} at most \texttt{nobj} objects of size
%   \texttt{size}, returning returns the number of objects read.

% \item[Fred] a person's name, e.g., there once was a dude named Fred
%   who separated usenix.sty from this file to allow for easy
%   inclusion.
% \end{description}

% \noindent
% The noindent at the start of this paragraph in its tex version makes
% it clear that it's a continuation of the preceding paragraph, as
% opposed to a new paragraph in its own right.


% \subsection{LaTeX-ing Your TeX File}
%-----------------------------------

% People often use \texttt{pdflatex} these days for creating pdf-s from
% tex files via the shell. And \texttt{bibtex}, of course. Works for us.

%-------------------------------------------------------------------------------
\section*{Acknowledgments}
%-------------------------------------------------------------------------------

Shih-Wei Li, John S. Koh, and Jason Nieh implemented HypSec, which serves as
the predecessor to SeKVM. Shih-Wei Li, Xupeng Li, Ronghui Gu, Jason Nieh, and
John Zhuang Hui implemented and formally verfiied SeKVM. Xuheng Li wrote a
helpful set of notes describing the boot and initialization process in SeKVM,
as well as the steps needed to build and run SeKVM~\cite{xuheng}. Jason Nieh and
Xuheng Li provided helpful comments on the project and helped with the design
on FlexAlloc. This paper was written as part of the class COMS E6118 Operating
Systems II at Columbia University.

%-------------------------------------------------------------------------------
\bibliographystyle{plain}
\bibliography{\jobname}

%%%%%%%%%%%%%%%%%%%%%%%%%%%%%%%%%%%%%%%%%%%%%%%%%%%%%%%%%%%%%%%%%%%%%%%%%%%%%%%%
\end{document}
%%%%%%%%%%%%%%%%%%%%%%%%%%%%%%%%%%%%%%%%%%%%%%%%%%%%%%%%%%%%%%%%%%%%%%%%%%%%%%%%

%%  LocalWords:  endnotes includegraphics fread ptr nobj noindent
%%  LocalWords:  pdflatex acks
